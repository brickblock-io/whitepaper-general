\documentclass[twoside,twocolumn,a4paper]{article}
\usepackage{EAAproceedings}
\begin{document}
\parskip 3pt  %Um Absatz genau zu definieren
% ****************  Begin of manuscript **********************
\Englishtitle{Brickblock General Whitepaper}

% Short title:
\Kolumnentitel{}

%\PACS{xx.xx.Nn, xx.xx.Nn}

% Authors' names, affiliations and addresses
% optionally grouped if with different institutions.
% First group:
\AuthorsI{Jakob Drzazga, Martin Mischke, Holger Schl�nzen, Philip Paetz}
\AddressI{info@brickblock.io}
%% Second group: (activate if applicable)
%\AuthorsII{version 1.02 in progress}
%\AddressII{Department of Proceedings,
%University of Bruckgrad,
%Bruckgrad, Country2.}
% Third group (activate if applicable)
% \AuthorsIII{}
% \AddressIII{}


% English Abstract:
\Englishabstract{Brickblock is building a new blockchain-based solution for investing in exchange-traded funds (ETFs), real estate funds (REFs), passive coin-traded funds (CTFs) and active coin managed funds (CMFs). Through the effective use of smart contracts, order and issuing fees can be reduced to a fraction of traditional costs. This will make investing in Brickblock more financially inclusive across all income classes. Artificial geographical trading restrictions, including the need for a bank account, can thus be eliminated. Counterparty risk can be reduced to mere minutes. Furthermore, we are introducing a new system of passively managed cryptocurrency baskets, which reduces the risks and high fees of actively managed coin funds by using liquidity providers, incentivized by arbitrage effects. Our system is based on an underlying rule-based asset allocation rather than high-risk contracts for difference (CFDs). It uses the asset-first principle, which incentivizes asset vendors to deliver assets before getting paid, thus further reducing investor risk.
}

\ScientificPaper

%%%%%%%%%%%%%%%%%%%%%%%%%%%%%%%%%%%%1
%%%%%%%%%%%%%%%%%%%%%%%%%%%%%%%%%%%%1
\section{Introduction - How does Brickblock work?}
At present, the crypto-economy is possibly one of the most volatile economies. Investors can make 10x gains in one month only to lose it all the next month. Thus far, there have only been a limited number of ways to diversify crypto-portfolios to mitigate these risks. Real diversification, across multiple asset classes, is essential for a well-balanced portfolio.

Brickblock introduces the first platform where users can seamlessly invest in real estate funds (REFs), exchange-traded funds (ETFs), passive coin-traded funds (CTF) and active coin managed funds (CMFs) through a streamlined process and with significantly lower costs than traditional investing.

Each fund on the Brickblock platform has its own denomination and its own "proof-of-asset" (PoA) token which, via established token exchange platforms, can be traded simpler, faster and cheaper than on conventional stock markets. Our asset-backed PoA tokens empower investors to hedge the risk of cryptocurrencies in real assets without needing to convert their cryptocurrency into a fiat currency. All dividends and coupons are automatically transferred to token-holders through self-executing smart contracts on the Ethereum network. The content of the smart contract and the PoA token itself is unalterable, cryptographically secured and visible to everyone on the blockchain.

To ensure the safety of the underlying assets represented by the PoA token, a digital trust fund holds the exact same amount of fund shares as tokens issued. This securities account cannot be claimed by Brickblock, the broker or anyone else, and is protected by strict laws regarding trust funds, even in the case of bankruptcy. Only token holders may reclaim their fund shares, at any time.

Brickblock's infrastructure will be implemented as a decentralized application (Dapp) and run on the Ethereum network. Broker-dealers and fund managers will be able to list their investment opportunities on the platform, after being thoroughly verified by Brickblock through e.g. proof of residence, credit reports and criminal record. Based on their personal risk/reward ratio, investors can then select an investment from the offered funds to add to their diversified portfolio. All fees, minimum investment pools, exchange rates, holding periods, net asset values, dividends or coupon payments will be listed clearly and thus, can easily be compared. All investment opportunities will be carefully validated and audited by independent parties (such as EY \cite{ref1}) to eliminate fraud.

\section {Market Analysis}
\subsection{Current Problems in the Market}
\subsubsection{High Volatility}
Cryptocurrencies are extremely volatile and, due to their limited acceptance in the traditional economy, investors have almost no options to disperse their risk.

Money has three important properties: It serves as a store of value, a medium of exchange and a unit of account. Currently, cryptocurrencies are primarily used as a store of value. The use of cryptocurrencies as a medium of exchange and as a unit of account is still in a very early stage. This means that there are limited options to escape the volatility of its value. At present, the only real way out is to reconvert into a fiat currency.
\subsubsection{Counterparty Risk}
When the counterparty does not possess the necessary funds or assets to settle a trade, this is known as  "counterparty risk", also "default risk".

In traditional asset trading, clearing and settlement processes take three days. On the first day, a broker makes a deal with the counterparty on the basis of trust. Based on this trust, the broker then makes countless other trades that all depend on the first trade actually being settled. On the second day, the clearing house logs the price, amount and payment method of the assets into its network. Only on the third day, two days after the actual trade, does the settlement and the exchange of delivery versus payment (DVP) actually occur.
\subsubsection{High Costs and Complexity in Conventional Stock Trading}
Trading fees, broker-assisted fees, account maintenance fees, account transfer fees, selling fees and commission fees at various stages are standard costs to consider when buying real-world assets through a retail broker. In addition to the complexity of determining which costs apply to which assets, minimum fees apply that make investing small amounts unprofitable and exclude lower income groups from participating in the market. Percentage-wise, a user that can only invest \$100 in an ETF pays a significantly higher fee than a professional trader investing \$100,000 - we think this is unfair and want to level the playing field here.
\subsubsection{Trading Restrictions}
It is highly complicated or even outright impossible for private investors to buy ETFs or REFs in some countries due to insufficient connectivity to worldwide exchanges. Investors need expensive bank or broker accounts and are severely limited in their freedom to trade by local governments. Furthermore, vastly different tax structures penalize worldwide investors.
\subsubsection{ No Safe Possibility of Crypto Basket Investments}
To date, no secure and easy way to invest in a crypto-basket with \textbf{full market exposure} in exchange for a single token exists. One must trust a fund manager to allocate the assets and not take advantage of their position; there is no legally binding contract between investors and fund managers. Indeed, there are no agreed-upon verification processes for fund managers or \textit{any} form of regulation. Investors are solely dependent on trusting that fund managers spend their funds how they said they would.
\subsection{How does Brickblock addresses current problems?}
\subsubsection{High Volatility}
By enabling the direct purchase of real-world assets, such as ETFs and REFs, with Ethereum, we are offering a simple way to hedge against volatile crypto-portfolios with low beta asset classes.
\begin{equation}
beta =  [Cov(r, K\textsubscript{m})]/ [StdDev(K\textsubscript{m}))]^2,
\label{eqm}
\end{equation}
\subsubsection{Counterparty Risk}
By using the blockchain for clearing and settlement processes, Brickblock reduces counterparty risk exposure from multiple days to less than 1 minute, or more precisely: to the length of a few Ethereum block confirmations.\cite{ref3}
\subsubsection{High Costs and Complexity in Conventional Stock Trading}
Through a high degree of automation, blockchain technology and the use of smart contracts, we are able to bypass a number of third parties, such as clearing houses and retail brokers. This allows us to reduce the fees associated with buying and selling real-world assets down to a fraction of traditional retail brokers. Depending on the country of purchase, the specific asset and the investment amount, Brickblock is able to offer up to a \textbf{50-fold discount} on the initial purchase price. If the asset is then subsequently sold on token exchanges, this discount can easily increase to 150-fold. An in-depth comparison of Brickblock fees vs. traditional broker fees is available on our blog.\cite{ref4}.
\subsubsection{Trading Restrictions}
No bank account is needed to trade global ETFs and REFs since no geographic borders apply. This enables Brickblock to offer the cheapest ETFs and REFs worldwide and the most economical way of order execution.
\subsubsection{ No Safe Possibility of Crypto Basket Investments}
For actively managed CMFs, Brickblock will enter legally binding contracts with fund managers, verify them thoroughly and implement a cryptographic audit structure. We, not the fund managers, will commission independent auditors to prevent any potential conflicts of interest between fund managers and auditors. 

For passively managed CTFs, rule-based fund indices in conjunction with the arbitrage phenomenon of foreign exchange (FX) markets are leveraged to obviate active fund managers and decentralize trades toward liquidity providers. Therefore, human error is minimized.
\subsection{Market Review}
In the past years, many token-funded startups have announced the development of financial tools to help traders and investors diversify within the crypto-currency market. In this section, we are going to examine several of these projects and highlight possible improvements. However, the purpose of this is not to devalue the competition, we merely seek to illustrate our motivation for developing our own solution, i.e. to address these problems. We have the highest respect for everyone that is sincerely trying to move the crypto-community forward and deeply believe that there is enough room for many innovative players in this field. 
\subsubsection{Taas}
Taas \cite{ref5} is a stand-alone, closed-end crypto-fund. Its whitepaper \cite{ref6} states that they have built a cryptographic audit system with Ambisafe \cite{ref7}. Users are able to track the trades of Taas on their website. We could not find any details about this system besides the trade records, which are made public on a subpage \cite{ref8}. From an investor's perspective, however, there is \textbf{no legal contract} between the user and Taas, which carries potential risks.\subsubsection{Melonport}
Melonport \cite{ref9} is an open source protocol (initially based on Ethereum and currently in development) that aims to simplify the process of developing portfolios of different cryptocurrencies for investors. Fund managers, on the other hand, can easily select what cryptocurrencies they want to buy. In comparison to Taas, the Melon Protocol, if executed safely in smart contracts, will eliminate the trust aspect. However, as far as we could interpret from the green paper \cite{ref10}, it is mostly based on contracts for difference (\textbf{CFDs}) for non-ERC20 tokens and coins. Since investments can only be made in Melons or ETH, users are heavily dependent on the stability of these two tokens. If, for example, a fund manager purchases a certain volume of Dash \cite{ref11}, no actual Dash tokens are being held by the user. From an investor's perspective, this means that there is a potential risk of not receiving the equivalent value of Dash if the Melons or ETH collateral cannot cover the gains or losses.
\subsubsection{Shapeshift's Prism}
Prism \cite{ref12} is also based on Ethereum. The user and a counterparty both deposit an equal amount of ETH into a Prism smart contract, which serves as collateral and is therefore also a contract for difference (\textbf{CFD}). The user bets, for example, on a bullish bitcoin\footnote{or any other cryptocurrency offered by Shapeshift's Prism} and the counterparty bets on a bearish bitcoin. This all works well if ETH is the users' domestic currency. However, if, as is the case for most investors, fiat currencies (like USD or EUR) are the users' domestic currency, then one could actually lose money. One could lose money even if they predicted the correct market movement of bitcoin, because the user's collateral will always be ETH regardless of the price. On the other hand, the user's profit is capped at 100\% gains, because Prism's collateral matches the user's. Thus, if one were to bet 1 ETH on Dash and Dash tripled in price, the maximum gain would not be 2 ETH but only 1 ETH.  
\subsubsection{Digix}
To the best of our knowledge, we are the first platform to build the necessary infrastructure to tokenize ETFs and REFs. The most comparable start-up to date, however, is Digix \cite{ref14}. Digix, which is currently in development, tokenizes a real-world asset: physical gold. After depositing ETH into a smart contract, a new DGX token is created and the asset vendor delivers the gold to a custodian. An auditor then regularly verifies whether the gold is still in possession of the custodian. In the last whitepaper \cite{ref15} we could find, it seems that Digix's asset vendors receive investors' funds before investors receive their gold. For example, if a great amount of gold is bought in a bearish crypto-market, the vendor either has to lend the money and wait until the audit is completed to receive investors' funds or the vendor receives the \textbf{payment in advance}, which constitutes a potential risk for the buyer. 
\subsubsection{Proof Suite}
Proof Suite \cite{ref16} is focusing on developing blockchain tools for tracking real-world assets. In their whitepaper, they present the benefits for companies and governments to replace existing systems with Proof Suite. They are trying to reduce bureaucratic procedures and make transactions more efficient. Nevertheless, they \textbf{rely on change occurring within companies and the legal acceptance} of their technology. Thus, if two parties were to trade real estate with Proof Suite and one of the parties decided to sell the same real estate to another party on the traditional market, legal problems occure.
\subsection{Difference to Existing Projects}
\subsubsection{Legal Enforceability of Claims}
Brickblock enters a \textbf{legally binding contract} with every fund manager and every broker-dealer. In contrast to direct and trust-based investments in fund managers, this adds an additional layer of security for investors. On the Brickblock platform, every fund manager has a direct liability to the investor, which is enforceable through legal action.
\subsubsection{Full Market Exposure}
We believe that CFDs and other derivatives rely too much on trust and moderate price trends. In the world of cryptocurrencies, where flash crashes \cite{ref17} and price movements of >200\% on a single day are a regular phenomenon, we consider CFDs no suitable instrument for mid- and long-term investments. Thus, \textbf{Brickblock only allows physical shares, commodities and currencies} instead of betting on derivatives.
\subsubsection{Escrow-backed Security for Investors and Brokers (asset-first principle)}
 Brickblock's smart contracts will only \textbf{release investors' funds to the broker-dealer after the broker-dealer transfers the assets} to the digital trust fund. This represents a new way of conducting clearing and settlement. Through automated smart contracts, the broker-dealer receives the investors' funds immediately after transmitting the assets. Initially, until new broker-dealers trust Brickblock's asset-first smart contracts, Brickblock will deposit the investor's funds in escrow as collateral. Therefore, if a smart contract does not activate, the broker-dealer can claim the escrow. 
\subsubsection{Connecting Old and New Economy}
Brickblock builds a bridge from the traditional investment world into the digital realm. We are leveraging existing infrastructure to enable faster adoption of this new system. All Brickblock transactions are legally accepted and enforceable in a court of law.

We believe a transition from old to new is best achieved by \textbf{building bridges}, not burning them.
\section{Who is Brickblock for?}
\subsection{ Private Investors}
Most importantly, Brickblock will help private investors diversify their portfolios beyond cryptocurrencies and tokens, reducing the overall risk. In addition to other benefits, this helps:
\begin{itemize}
  \item Significantly lower the costs of investing in REFs, ETFs, CMFs and CTFs through cutting out the middlemen and pooling investment volume,
  \item Create steady returns in the form of dividends and coupons,
  \item Hedge the systemic risk of a heated market,
  \item Clarify fees, tracking errors and liquidity,
  \item Minimize bureaucratic overhead,
  \item Empower \textbf{everyone} to invest directly in global funds in every market, regardless of where funds or investors live.
\end{itemize}
\subsection{Institutional Investors}
Brickblock will help institutional investors invest their funds in a diversified digital currency portfolio without having to worry about holding multiple wallets and handling a multitude of exchange platforms. 
\subsection{Fund Managers}
\subsubsection{Real Estate Fund Managers}
Real estate fund Managers are responsible for a variety of tasks. Some of the most important of which are: supervising real estate acquisition; evaluating consultants, appraisers and property managers; designing financial models and formulating asset allocation strategies. 

In the traditional system, fund managers must pay enormous provisions to banks to sell their funds to customers, which is only possible if the fund is large enough. For smaller funds, which cannot afford to pay for distribution, their only chance is to maintain all investor relationships themselves. This represents a significant competitive disadvantage, since this costs significantly more time, which then cannot be spent increasing the rentability of the fund. Brickblock enables immediate global distribution, reducing fund costs, increasing profitability, lowering fund managers' workload and overall, leveling the playing field.
\subsubsection{Coin Managed Fund Manager}
Coin managed fund managers, either professional or social, gain a platform for advertising their past performance and attracting new investors. Coin managed fund managers can freely set their management fee structure and have Brickblock as a legal and trustworthy entity whom they can contract. Brickblock is interested in productive professional relationships and will help fund managers establish the initial management structure. Furthermore, Brickblock will work with third parties to pro-actively reduce potential legal and cyber security risks associated with the responsibility of being a fund manager.
\subsection{Broker-Dealers}
Broker-Dealers, sometimes also referred to as "market makers", are liquidity providers at traditional stock exchanges. Most large institutional ETF block trades do not take place at exchanges, but over-the-counter (OTC) and are not visible to the public. This is especially the case for illiquid exotic ETF underlyings. In contrast to the immediate execution on exchanges, broker-dealers prefer to be asked for the price of specific products and quantities. They prepare all background hedging and internal risk exposure analyses before quoting and executing an order. The quoted price is often better than the bid ask offer on exchanges.

The digital trust "request for quote" (RFQ) issued by the smart contract can be parsed and tokenized in a variety of industry standard formats, such as FIX \cite{ref18} ensuring compatibility with existing processes and legacy integrations. Brickblock acts as a single counterpart against the broker-dealer. This simplifies compliance and set-up processes for broker-dealers. Brickblock will assist broker-dealers with a dedicated e-wallet application programming interface (API) solution.

The focus is to radically reduce operational costs and enable delivery versus payment (DvP) support, including crypto-payments. Compliance costs (e.g. maintenance of segregated accounts), onboarding, settlement and know your customer (KYC) processes are also included, where applicable. The key is that broker-dealers are free to focus on the core business. Broker-dealers profit by increasing their trading volume of securities, and Brickblock improves the liquidity of the crypto-economy. 
\subsection{ Commercial Paper Issuers}

Brickblock will help issuers enter the crypto-economy by integrating existing asset management legacy infrastructure and messaging standards for both issuance and distribution. There is a strong incentive in conventional paper issuance to move toward lower cost distribution models and reach international marketplaces.  

% Syntax is:
%\Figure{parameter1}{parameter2}{parameter3}{parameter4}, where
%     parameter1 = label which can be used in \ref{} commands
%     parameter2 = caption
%     parameter3 = height of figure frame box, in millimeters. You will need to choose a suitable value.
%     parameter4 = figure insertion command. Two different examples are given.
\FIGURE{ETFREF}{ETF/REF: User's buying process}
        {100}{  \put(6,6)  {  \includegraphics[width=150mm]{ETFREF}}}
\section {The Platform} 
\subsection {Tradeable assets on Brickblock} 
After choosing a fund, the user deposits the payment into a smart contract. The smart contract controls all fees, minimum investment pools, exchange rates, holding periods, net asset values, dividends or coupon payments. Therefore, depending on the asset class, there are three different scenarios.

\subsubsection {Real Estate Funds (REFs)} 
Investing in real estate comes with major diversification potential, depending on both the users' foreign exchange (FX) exposure and tax domicile. Our platform will enable fund managers to list real estate projects with high potential and help users to easily discover interesting opportunities. Each project will be carefully audited by independent parties to minimize fraud.

Real estate funds (REFs) are actively managed by a fund manager. Depending on its underlying focus, the fund manager acquires a certain volume of real estate in certain cities, countries or continents across the globe.

Real estate has two growth functions. First, there is the value of the real estate itself, which may potentially grow over time. Second, real estate produces a steady income in the form of rent, which is distributed to its shareholders via smart contracts or re-invested in new real estate, depending on user preference. By choosing a fund, users can influence the frequency of distributions and other factors such as costs and fees, area of investment, net asset value (NAV) and minimum investments.

Brickblock will start with European REFs and subsequently extend into more locations. After the processes for REFs are established, we will further broaden the investment scope to also include real estate investment trusts (REITs) and real estate crowdfunding.
\subsubsection {Exchange-traded Funds (ETFs)} 
Exchange-traded funds are an interesting store of value for many investors. Unlike actively managed funds, the fees and costs are significantly lower since ETFs passively track rule-based indices like the S\&P 500, the Nikkei or the Dax 30. Exchange-traded funds can also track commodities like gold and silver and have the advantage of being offered at near-wholesale prices without minimum purchase amounts. Exchange-traded funds outperform the returns of most actively managed funds that invest in company shares \cite{ref19} and enable investors to diversify their portfolio, even with very little capital. By choosing their preferred ETFs, investors can actively decide in which markets to invest and spread their risk across different geographic locations and industries.

Brickblock will start with the ETFs that track global indices and will then inquire after the community as to which ETFs to implement next. Brickblock will establish connections with and request quotes from different broker-dealers based on past pricing performance. This guarantees the best execution of orders.
\subsubsection {Coin Managed Funds (CMF)} 
Actively managed CMFs provide projects like Taas \cite{ref5} the opportunity to offer their funds to the public. Fund managers can manage investors' funds by selecting specific cryptocurrencies, day-trading, taking profits out of diversifying portfolios and the active distribution of funds. Conversely, users can decide which strategies and fund managers to trust. Each project will be carefully audited by independent parties to significantly minimize the risk of fraud.

        
\subsubsection {Coin-traded Funds (CTF)} 
Brickblock offers passively managed CTFs and tracking rule-based indices designed by fund managers and community members. One of these could be a Top 10 cryptocurrencies market cap CTF. The index would include the Top 10 cryptocurrencies weighted by market capitalization. Since no passive CTF currently exists, we are introducing a system based on passive ETFs, as explained in section \ref {CTFtext}

Further, a separate non-profit organization (NPO) is set up; leading members of the crypto-finance community are invited to join us in improving the system and standardizing processes.

\subsection {Technical Setup} 
The following framework describes the underlying technical processes. Our platform will be developed as a Dapp on top of the Ethereum blockchain. The Dapp will guide users through all of the investment steps. All Ethereum transactions relevant to Brickblock will additionally be backed up on our own blockchain, with only one public node for security reasons.
\subsubsection {Real Estate Funds (REFs) and Exchange-traded Funds (ETFs)
} 
Figure \ref{ETFREF} displays the detailed process of investing in real-world assets such as REFs and ETFs. 
        
After choosing an asset class, either REF or ETF, users can select a fund based on their investment preferences.  Users will be informed about fees, frequency of distributions, areas of investment, current net asset value (NAV), minimum investment volume, performance or track record.

Once users select a fund, they will be asked to deposit the desired investment amount into a smart contract. A minimal creation size helps reduce transaction costs and decouples the PoA tokens from the nominal value of the fund. After reaching the minimal creation size, the broker-dealer places a fund order. The purchased securities are then deposited into a digital trust fund by the broker-dealer. Simultaneously, the custodian of the digital trust fund issues a cryptographically signed electronic document confirming the receipt of assets. This document is verified by the smart contract to ensure that the broker-dealer's order matches the user's order. If the orders match, the smart contract treats this as proof that the broker-dealer's obligations are properly fulfilled. The smart contract subsequently releases the user's deposited funds to the broker/dealer, and a PoA token to the user. This token serves as a proof of assets.

The digital trust fund is only allowed to hold the securities for the actual beneficiary of the PoA token. If PoA token holders want to sell their assets, they can sell the PoA tokens on token exchanges like EtherDelta, iDex, 0x or withdraw the securities from the digital trust fund at any time after going through a KYC process provided by the bank. If dividends or coupons are issued by the fund, the custodian of the digital trust fund will send these proceeds to the self-executing smart contract, which will automatically allocate them to the respective token holders.


\subsubsection {Active Coin Managed Funds (CMFs)}
Coin managed fund managers will be able to offer funding campaigns on our platform and manage funds by investing in different cryptocurrencies. A cryptographic audit infrastructure will thoroughly verify every new individual through, inter alia, proof of residence, credit report and criminal record. Furthermore, all trades will be recorded and stored safely. 
Users can choose between:
\begin{enumerate}
  \item Fund managers who use \textbf{secured accounts}

Users' funds are transferred to secured accounts at trustful exchanges, where withdrawal functions are blocked. Users then receive PoA tokens in exchange for their funds. Through the use of secured accounts, the risk of fraud and private key loss are minimized.
\item Fund managers who use \textbf{unsecured accounts}

For digital assets that are not traded over common exchanges, the fund manager receives the users' funds from the smart contract into an unsecured account, controlled by the fund manager. Users again receive a PoA token in exchange. In this case, the legal contract between Brickblock and the fund manager protects users from fraud. In case of any irregularities, Brickblock is able to enforce users' claims through legal action. 
\end{enumerate}

The crypto-backed PoA token represents the invested funds actively managed by a fund manager. The fund manager diversifies the received funds into various other cryptocurrencies and tokens.

Fund managers can choose from different fee structures: percentage of gains, percentage of funds managed or a fixed fee. Users can then choose a fund manager based on compliance with their desired fee structure and risk/reward ratio.

After a predetermined trading time window, funds are transferred back to the smart contract, which then forwards them to the users' Ethereum wallets, minus the fund manager's fees.

\subsubsection {Passive Coin-traded Funds (CTFs)}
\label{CTFtext}
The CTF concept maps the efficient ETF mechanism onto the blockchain. A CTF is a smart contract that tracks an underlying index of crypto-assets as closely as possible. The holdings of the digital trust fund, in the form of crypto-wallets, are stored in cold storage by a custodian. This is necessary because an Ethereum smart contract currently cannot reliably handle automated transactions with other blockchains. A CTF provides both a creation and a redemption basket, necessary for creating and redeeming tokens. The composition of these baskets helps rebalance the funds' holdings according to the underlying index.

The user first decides whether to buy the token via an exchange trade or using the creation mechanism.
\paragraph {The Trading Process}
The investment process for passive CTFs is similar to investing in REFs and ETFs. Figure \ref {CTFexchange} illustrates the process flow in detail. Users can choose between different indexing methods designed by fund managers. The index is the basis for the CTF composition and determines risk and performance. Brickblock will support users by providing CTF factsheets with holdings, past performance and tracking quality of the CTF.

\begin{figure}[htbp]
\centering
\includegraphics[width=8cm]{CTFexchange}
\caption{CTF: User's buying process}
\label{CTFexchange}
\end{figure}

\paragraph {The Creation Process}
To prevent trading and therefore, trading fees inside the CTF, we utilize liquidity providers (LPs). The LPs hold PoA tokens in their account and sell them on exchanges. Furthermore, LPs can create and redeem tokens from the fund.  Figure \ref {CTFcreation} illustrates the process flow in detail. Tokens are created by sending the components specified in the creation file to the custodian and the correspondent wallet addresses. If the correct basket is delivered to the wallets, an automated message is sent to the CTF smart contract and the PoA token is released. Liquidity providers help minimize the running costs of trading and rebalancing inside the fund and take advantage of arbitrage effects on the market.

If the CTF fund accepts cash redemptions (paying out ETH for PoA tokens), existing PoA token holders are charged for the resulting trading costs and thus, penalized. Therefore, LPs receive the assets of the redemption basket in return for the PoA token and must manage the trading themselves. 

Rather than buying PoA tokens via an exchange, users are allowed to use the creation mechanism as long as they hold all necessary assets. However, for unprofessional users, it is more convenient to buy already created tokens on token exchanges.

\begin{figure}[htbp]
\centering
\includegraphics[width=8cm]{CTFcreation}
\caption{CTF: User's creation process}
\label{CTFcreation}
\end{figure}

All CTF-relevant entities are listed below:
\paragraph {Users/Investors}

select in which CTF to invest and can trade PoA tokens on token exchanges. Moreover, users are allowed to create tokens by sending a "creation unit" to the CTF smart contract and receiving PoA tokens of the same value in return. 

\paragraph {Liquidity Providers (LPs)}
are at the center of the creation and redemption process of PoA tokens. They have the right to redeem PoA tokens against the fund's holdings. To reduce trading within the CTF and save on transaction fees, the redemption units are consolidated into blocks of 5,000 tokens. Like users, LPs can create tokens by delivering the creation basket to the fund (the custodian wallets). In exchange, they receive the same value in PoA tokens. 

The LPs act as market makers on exchanges. To properly fulfill this role and react in a timely manner, the LPs hold PoA tokens in their own accounts. LPs act as a buffer between the investor and the fund. The number of LPs for one CTF is unlimited.


\paragraph {Coin-traded Fund Smart Contracts}
seek to replicate the index as closely as possible. They publish the creation and redemption baskets on a daily basis. The baskets describe the assets necessary to create new tokens, or those which the LPs receive when a PoA token is redeemed. The creation and redemption baskets are designed to adjust the assets to the underlying index. 

\paragraph {Creation/Redemption Files}

describe the creation basket. It is the composition of the coins or tokens for creating new PoA Tokens. The creation/redemption file is released by the CTF smart contract.
\paragraph {Underlying Indices}

are the heart of the CTF. The CTF tries to track the index as closely as possible without tracking errors or differences. The methodology and rules of the index are designed by fund managers or index providers. These rules can be market-cap oriented, factor or momentum strategies. Every blockchain asset can be included within the index. 
\paragraph {Custodian}

The custodian holds the different crypto-wallets in cold storage. The transactions are automatically sent to the CTF smart contracts, which confirm receipt of the correct creation basket.

\section {Tokens} 
There are three types of tokens, all of which implementing the ERC20 token standard \cite{ref21}. Detailed information about volume, distribution mechanism and the price of tokens will be released separately ahead of the contribution period.

\subsection {Brickblock Tokens} 
Brickblock tokens will only be released during the contribution period in exchange for ETH and Bitcoin. If the Brickblock tokens are stored in a special smart contract, the Brickblock token holder will receive a certain amount of access tokens per week until the Brickblock tokens are withdrawn from the smart contract. 
\subsection {Access Tokens} 

Broker-dealers and fund managers need access tokens to list their fund on the Brickblock platform and to pay market-determined Brickblock fees when REFs, ETFs, CMFs or CTFs are sold. The access token is then burned. Brickblock will determine the amount of access tokens based on supply and demand. 

Brickblock is incentivized to establish the amount of access tokens needed in such a way so as to positively influence the usage of the platform, while still making a profit. All access token holders can act as competitors to Brickblock and sell their access tokens to broker-dealers and fund managers on the open market. Brickblock cannot increase fees without risking a decline in the number of trades conducted on the platform and cannot decrease fees without reducing its own profit.


\subsection {Proof-of-Assets (PoA) Tokens } 
For ETFs and REFs, PoA tokens represent real-world assets in the form of securities. For CMFs and CTFs, they represent a claim to coin funds on a secured trading account or in the custodians' wallets. Users receive a PoA token in return for every fund in which they invest, which represents a legally enforceable claim to the underlying assets. 

\section {Conclusion and Vision} 
Brickblock is an inclusive investment platform that empowers people from all income classes to diversify their crypto-portfolio with real-world assets like ETFs and REFs. We will provide legal frameworks to directly link these assets to our PoA tokens. Furthermore, Brickblock will lead the development of passively managed CTFs, as well as offer actively managed CMFs.

At Brickblock, we strongly believe that digital currencies are the future. The monetary system and the financial services industry are extremely old-fashioned and ripe for disruption \cite{ref22}. It still takes 5 days for an international money transfer to crawl through the SWIFT network, originally created in 1973 \cite{ref23}, and it still takes 2 days to complete clearing and settlement processes of asset-transfers. Present technology is far more advanced than the heavily outdated regulatory frameworks of sluggish banks. 

Various members of the Brickblock team have spent several years working professionally in conventional asset management and understand the processes and issues of trading and settlement. The asset management and global custody ecosystem is highly complex, extremely cost intensive in a multitude of ways (execution, reconciliation, allocation settlement etc.) and exceptionally exclusive.

Our mission is to change that.

We believe that by including more people in the global economy and enabling them to invest their money however they like, everyone will win: investors will pay significantly lower fees, have more options to hedge volatility and risks that accompany the digital economy and bypass unfair local jurisdictions. Furthermore, trusted brokers/dealers and fund managers will obtain an entirely new group of financiers, and underfinanced companies and industries will gain access to new capital.

Incumbent banks neither have the ability nor the incentive to free themselves from the burden of the status quo. They lack vision. Any attempt to innovate would meet with disapproval from the board. 

So it is up to us, the crypto-community, to use the aspiring blockchain technology to eliminate the faults of the old economy and to develop tools which make it simple, affordable and safe to access the instruments currently reserved for the financial class.

It is up to us, to create a system that gives everyone access to real-world assets and to participate in profiting from global economic progress. No matter where and no matter if you have a bank account or not.

It is up to us, to make trading, clearing and settlement with any kind of assets as easy and secure as transferring a bitcoin. 
 
If you are interested in shaping the future of investing with us: We are always looking for talented people to help us in making the crypto-world better, safer and more inclusive. Get in touch.


\section {Glossary}

\begin{description}  
\item  \textbf{Access Token:} An Access Token is used by broker-dealers and fund managers to pay the platform service fees.
\item \textbf{Brickblock Token:} A Brickblock Token is received in the contribution period and is the only way to generate access tokens.
\item \textbf{Broker-dealer:} A Broker-dealer is an entity that trades securities for its own account or on behalf of its customers. Broker-dealers are often market-makers and authorized participants in the ETF market; they act as liquidity providers on exchanges.
\item \textbf{CFD:} A "contract for difference" allows traders to speculate on the movement of an asset price without owning the underlying. A buyer and seller devise a contract to exchange the difference in the current value of the underlyings.
\item \textbf{CMF:} A "coin managed fund" is the crypto-equivalent of an actively managed investment fund, where the portfolio manager chooses the fund's investments.
\item \textbf{CTF:} A "coin-traded fund" is the crypto-equivalent of an ETF; it is a passively managed basket of different cryptocurrencies. 
\item \textbf{CTF Constituents:} The constituents of a CTF are the holdings of the fund.
\item \textbf{Creation Basket:} A creation basket is the exact list of assets that need to be sent to the CTF to create a PoA Token.
\item \textbf{Creation File:} The Creation File contains the details of the creation basket.
\item \textbf{Custodian:} A custodian holds the assets for a fund. The assets (cash and securities) of a fund must be maintained in the cash/securities account opened in the name of that fund. 
\item \textbf{Dapp:} A "decentralized app" is an application that runs predominantly on the blockchain.
\item \textbf{DVP:} "Delivery versus payment" is a securities settlement procedure in which the transfer of the securities and payments occur simultaneously and no party holds both at the same time.
\item \textbf{ERC20 Token:} The ERC20 token is a widely tradable token that implements the ERC20 standard. \cite{ref21}
\item \textbf{ETF:} An "exchange-traded fund" is an investment fund traded on exchange, which passively tracks a rule-based index.
\item \textbf{LP:} A "liquidity provider" provides liquidity for proof-of-asset tokens on exchanges. The LP can create and redeem CTF tokens against the portfolio holdings.
\item \textbf{NAV:} The "net asset value" is the current value of the fund holdings divided by the number of shares. The NAV therefore, is the price of one of the fund's shares. The NAV is calculated on a daily basis at a fixed time.
\item \textbf{OTC:} "Over-the-counter", within a trading context, means that the trade is not executed on an exchange, but privately in a dealer network or over the phone.
\item \textbf{PoA Token:} A "Proof-of-Asset" Token represents a real-world asset in the form of securities or, in the case of CTFs, the right to Coin funds on a certain secured trading account
\item \textbf{REF:} A "real estate fund" invests directly in commercial and residential property. Most REFs focus on a specific type of assets (e.g. luxury housing) or region (e.g. Europe).
\item \textbf{REIT:} A "real estate investment trust" is a company that, in most cases, owns and operates income-producing real estate assets. Some REITs provide loans to the owners and operators of real estate.
\item \textbf{RFQ:} "Request for quote"  describes the process of sending standardized quote requests to select brokerage firms. The quote is constantly updated and the processing can be fully automated.
\item \textbf{Smart Contract:} Smart contracts are computer protocols that have fixed if-then relations, therefore facilitating contract design and enforcement.
\end{description}
%%%%%%%%%%%%%%%%%%%%%%%%%%%%%%%%%%%%
%%%%%%%%%%%%%%%%%%%%%%%%%%%%%%%%%%%%
\begin{thebibliography}{1}

\bibitem{ref1}
 https://www.ey.com/ 
\bibitem{ref3}
 https://etherscan.io/chart/blocktime 
\bibitem{ref4}
https://medium.com/@Brickblock/brickblock-will-lead-the-way-in-making-investing-more-financially-inclusive-around-the-world-97952f80925d 
\bibitem{ref5}
https://taas.fund 
\bibitem{ref6}
https://taas.fund/media/whitepaper.pdf 
\bibitem{ref7}
 https://www.ambisafe.co 
\bibitem{ref8}
https://ca.taas.fund/wallets/dashboard
\bibitem{ref9}
https://www.melonport.com 
\bibitem{ref10}
https://github.com/melonproject/greenpaper/blob/master/melonprotocol.pdf 
\bibitem{ref11}
 https://www.dash.org/blockchain-explorers/ 
\bibitem{ref12}
https://info.shapeshift.io/blog/2017/05/21/introducing-prism-worlds-first-trustless-portfolio-market-platform 
\bibitem{ref14}
 https://www.digix.io 
 \bibitem{ref15}
https://dgx.io/whitepaper.pdf 
\bibitem{ref16}
https://www.proofsuite.com 
\bibitem{ref17}
https://blogs.wsj.com/moneybeat/2017/06/23/ethereums-flash-crash-shows-hazards-of-trading-cryptocurrencies/ 
\bibitem{ref18}
http://www.fixtradingcommunity.org/pg/structure/tech-specs 
\bibitem{ref19}
https://www.justetf.com/uk/news/passive-investing/the-proof-that-active-managers-cannot-beat-the-market.html 
\bibitem{ref21}
https://github.com/ethereum/EIPs/issues/20 
\bibitem{ref22}
https://www.forbes.com/sites/ciocentral/2017/02/24/is-the-financial-services-industry-ripe-for-disruption/\#e474e5378af2
\bibitem{ref23}
 https://www.swift.com/about-us/history 

\end{thebibliography}



DISCLAIMER: This Brickblock whitepaper is for information purposes only and is subject to change. Brickblock does not guarantee the accuracy of or the conclusions reached in this white paper, and this whitepaper is provided "as is". Brickblock does not make and expressly disclaims all representations and warranties, express, implied, statutory or otherwise, whatsoever, including, but not limited to: (i) warranties of merchantability, fitness for a particular purpose, suitability, usage, title or non-infringement; (ii) that the contents of this white paper are free from error; and (iii) that such contents will not infringe third-party rights. Brickblock and its affiliates shall have no liability for damages of any kind arising out of the use, reference to, or reliance on this white paper or any of the content contained herein, even if advised of the possibility of such damages. In no event will Brickblock or its affiliates be liable to any person or entity for any damages, losses, liabilities, costs or expenses of any kind, whether direct or indirect, consequential, compensatory, incidental, actual, exemplary, punitive or special for the use of, reference to, or reliance on this whitepaper or any of the content contained herein, including, without limitation, any loss of business, revenues, profits, data, use, goodwill or other intangible losses.

\end{document}

